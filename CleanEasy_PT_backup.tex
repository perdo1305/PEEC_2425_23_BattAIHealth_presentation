%----------------------------------------------------------------------------------------
%    PACKAGES AND THEMES
%----------------------------------------------------------------------------------------
\documentclass[aspectratio=169,xcolor=dvipsnames]{beamer}
\makeatletter
\def\input@path{{theme/}}
\makeatother
\usetheme{CleanEasy}
\usepackage{lmodern}
\usepackage[T1]{fontenc}
\usepackage[portuguese]{babel}
\usepackage{fix-cm}
\usepackage{amsmath}
\usepackage{mathtools}
\usepackage{listings}
\usepackage{xcolor}
\usepackage{hyperref}
\usepackage{graphicx} % Allows including images
\usepackage{booktabs} % Allows the use of \toprule, \midrule and \bottomrule in tables
\usepackag    \draw[thick, <->] (2.2,1.5) -- (3.8,2);
    
    \node at (3,1) {Trade-off};
  \end{tikzpicture}
\end{frame}z}
\usetikzlibrary{positioning, shapes, arrows, calc, decorations.pathreplacing, arrows.meta, backgrounds, patterns, overlay-beamer-styles}
\usepackage{etoolbox}
\usepackage{animate}

%----------------------------------------------------------------------------------------
%    LAYOUT CONFIGURATION
%----------------------------------------------------------------------------------------
\input{configs/configs}
%----------------------------------------------------------------------------------------
%    TITLE PAGE
%----------------------------------------------------------------------------------------
%---------------------------------------------


\title[BattAIHealth]{BattAIHealth: AI for Battery Health Monitoring}

\author[Pedro Ferreira]{Pedro André Silva Ferreira}

\institute{Engenharia Eletrotécnica e de Computadores\\Ramo de Eletrónica e Computadores }

\date{21 Julho 2025}
% Define positions for logos on title page
\titlegraphic{
  \begin{tikzpicture}[remember picture, overlay]
    % ESTG Logo
    \node[anchor=north east, xshift=-0.8cm, yshift=-0.3cm] at (current page.north east) {
      \includegraphics[height=1.5cm]{logos/estg_h.pdf}
    };
  \end{tikzpicture}
}


%----------------------------------------------------------------------------------------


\begin{document}

\begin{frame}[plain]
  \titlepage
\end{frame}

\begin{frame}[plain]{Conteúdos}
  \tableofcontents
\end{frame}

\section{Introdução}
\begin{frame}{Motivação}
  \begin{block}{Contexto}
    \begin{itemize}
      \item Veículos elétricos e comboios eletrificados dependem de monitorização precisa da saúde das baterias
      \item Parâmetros críticos: SOC, SOH e RUL
      \item Processos químicos complexos que mudam com temperatura, padrões de uso e envelhecimento
    \end{itemize}
  \end{block}
  
  \begin{alertblock}{Limitações dos Métodos Atuais}
    \begin{itemize}
      \item Métodos tradicionais (coulomb counting, filtros de Kalman) têm limitações
      \item Funcionam bem em ambientes controlados, mas falham em condições reais
      \item Acumulação de erros ao longo do tempo
    \end{itemize}
  \end{alertblock}
\end{frame}

\section{Fundamentos Teóricos}
\begin{frame}{Conceitos Fundamentais - Parâmetros das Baterias}
  \begin{block}{Estado de Carga (SOC)}
    $$SOC = \frac{\text{Carga Remanescente}}{\text{Capacidade Máxima}} \times 100\%$$
    \begin{itemize}
      \item Estimativa aproximada devido à complexidade química
      \item Não-linearidade causada por degradação dos eletrodos
    \end{itemize}
  \end{block}
  
  \begin{block}{Estado de Saúde (SOH)}
    $$SOH = \frac{\text{Capacidade Máxima Atual}}{\text{Capacidade Máxima Original}} \times 100\%$$
    \begin{itemize}
      \item Fadiga progressiva dos materiais
      \item Diminuição da área superficial ativa
    \end{itemize}
  \end{block}
\end{frame}

\begin{frame}{Degradação e Envelhecimento das Baterias}
  \begin{alertblock}{Mecanismos de Degradação}
    \begin{itemize}
      \item \textbf{Formação SEI}: Crescimento da interface sólido-eletrólito
      \item \textbf{Perda de Lítio}: Redução do lítio ativo disponível
      \item \textbf{Degradação estrutural}: Mudanças na estrutura dos eletrodos
      \item \textbf{Impedância}: Aumento da resistência interna
    \end{itemize}
  \end{alertblock}
  
  \vspace{0.3cm}
  
  \begin{exampleblock}{Fatores que Afetam a Degradação}
    \begin{itemize}
      \item Temperatura de operação
      \item Profundidade de descarga
      \item Taxa de carga/descarga
      \item Número de ciclos
    \end{itemize}
  \end{exampleblock}
\end{frame}

\begin{frame}{Análise de Séries Temporais}
  \begin{block}{Componentes das Séries Temporais}
    \begin{enumerate}
      \item \textbf{Tendência}: Direção a longo prazo (degradação da capacidade)
      \item \textbf{Sazonalidade}: Padrões cíclicos (ciclos diários de uso)
      \item \textbf{Irregular/Ruído}: Variações aleatórias e erros de medição
    \end{enumerate}
  \end{block}
  
  \begin{exampleblock}{Análise Espectral com FFT}
    \begin{itemize}
      \item Identifica padrões periódicos ocultos
      \item Frequências de carga-descarga
      \item Padrões de uso diários/sazonais
      \item Separação de ruído do sinal real
    \end{itemize}
  \end{exampleblock}
\end{frame}

\section{Introdução}
\begin{frame}{Motivação}
  \begin{block}{Contexto}
    \begin{itemize}
      \item Veículos elétricos e comboios eletrificados dependem de monitorização precisa da saúde das baterias
      \item Parâmetros críticos: SOC, SOH e RUL
      \item Processos químicos complexos que mudam com temperatura, padrões de uso e envelhecimento
    \end{itemize}
  \end{block}
  
  \begin{alertblock}{Limitações dos Métodos Atuais}
    \begin{itemize}
      \item Métodos tradicionais (coulomb counting, filtros de Kalman) têm limitações
      \item Funcionam bem em ambientes controlados, mas falham em condições reais
      \item Acumulação de erros ao longo do tempo
    \end{itemize}
  \end{alertblock}
\end{frame}

\begin{frame}{Objetivos do Trabalho}
  \begin{block}{Objetivo Principal}
    Desenvolver métodos baseados em IA para prever SOC, SOH e RUL simultaneamente e com precisão
  \end{block}
  
  \vspace{0.3cm}
  
  \begin{block}{Parâmetros a Estimar}
    \begin{itemize}
      \item \textbf{Estado de Carga (SOC)}: Energia remanescente na bateria
      \item \textbf{Estado de Saúde (SOH)}: Capacidade atual vs. capacidade original
      \item \textbf{Vida Útil Remanescente (RUL)}: Ciclos até degradação crítica
    \end{itemize}
  \end{block}
  
  \begin{exampleblock}{Contribuições}
    \begin{itemize}
      \item Comparação entre abordagens tradicionais e baseadas em IA
      \item Adaptação da arquitetura TimesNet para dados de bateria
      \item Análise de trade-offs na predição multi-paramétrica
    \end{itemize}
  \end{exampleblock}
\end{frame}

\section{Fundamentos Teóricos}
\begin{frame}{Conceitos Fundamentais - Parâmetros das Baterias}
  \begin{block}{Estado de Carga (SOC)}
    $$SOC = \frac{\text{Carga Remanescente}}{\text{Capacidade Máxima}} \times 100\%$$
    \begin{itemize}
      \item Estimativa aproximada devido à complexidade química
      \item Não-linearidade causada por degradação dos eletrodos
    \end{itemize}
  \end{block}
  
  \begin{block}{Estado de Saúde (SOH)}
    $$SOH = \frac{\text{Capacidade Máxima Atual}}{\text{Capacidade Máxima Original}} \times 100\%$$
    \begin{itemize}
      \item Fadiga progressiva dos materiais
      \item Diminuição da área superficial ativa
    \end{itemize}
  \end{block}
\end{frame}

\section{Estado da Arte}
\begin{frame}{Métodos Tradicionais}
  \begin{columns}
    \begin{column}{0.48\textwidth}
      \textbf{Modelos de Circuito Equivalente}:
      \begin{itemize}
        \item Modelo de Thévenin
        \item Modelo PNGV
        \item Representam comportamento eletroquímico
      \end{itemize}
      
      \vspace{0.3cm}
      
      \textbf{Coulomb Counting}:
      \begin{itemize}
        \item Integração da corrente
        \item Acumulação de erros
        \item Limitações de precisão
      \end{itemize}
    \end{column}
    \begin{column}{0.48\textwidth}
      \textbf{Filtros de Kalman}:
      \begin{itemize}
        \item Filtro de Kalman Estendido (EKF)
        \item Estimativa de estado
        \item Fusão de sensores
      \end{itemize}
      
      \begin{alertblock}{Limitações}
        \begin{itemize}
          \item Não-linearidade
          \item Variabilidade ambiental
          \item Envelhecimento não modelado
        \end{itemize}
      \end{alertblock}
    \end{column}
  \end{columns}
\end{frame}

\begin{frame}{Métodos Baseados em IA}
  \begin{block}{Redes Neurais Recorrentes}
    \begin{itemize}
      \item RNNs básicas com memória limitada
      \item LSTMs para dependências temporais longas
      \item Melhor modelação da degradação da bateria
    \end{itemize}
  \end{block}
  
  \vspace{0.3cm}
  
  \begin{block}{Arquiteturas Avançadas}
    \begin{itemize}
      \item \textbf{Transformers}: Mecanismos de atenção para sequências longas
      \item \textbf{Mixture of Experts (MoE)}: Especialização para diferentes padrões
      \item \textbf{CNNs}: Processamento de padrões espaciais/temporais
    \end{itemize}
  \end{block}
  
  \begin{exampleblock}{Vantagens da IA}
    \begin{itemize}
      \item Captura de padrões complexos e não-lineares
      \item Adaptação a diferentes condições
      \item Aprendizagem de relações ocultas
    \end{itemize}
  \end{exampleblock}
\end{frame}

\section{Desenvolvimento}
\begin{frame}{Modelação MATLAB e Simulação}
  \begin{block}{Implementação EKF em Simulink}
    \begin{itemize}
      \item Framework completo de simulação
      \item Estimativa de SOC usando filtro de Kalman estendido
      \item Integração com modelo Batemo INR21700-p45b
    \end{itemize}
  \end{block}
  
  \begin{exampleblock}{Modelo Baseado na Física}
    \begin{itemize}
      \item Maior precisão que modelos de bateria padrão
      \item Parâmetros de envelhecimento e configuração da bateria
      \item Validação dos métodos tradicionais
    \end{itemize}
  \end{exampleblock}
\end{frame}

\begin{frame}{Arquiteturas Neurais Baseline}
  \begin{columns}
    \begin{column}{0.48\textwidth}
      \textbf{Transformer}
      \begin{itemize}
        \item Mecanismos de auto-atenção
        \item Excelente para sequências longas
        \item Alta precisão mas computacionalmente intensivo
      \end{itemize}
      
      \vspace{0.5cm}
      
      \centering
      \begin{tikzpicture}[scale=0.6]
        \draw[thick] (0,0) rectangle (2,0.8) node[pos=.5] {\scriptsize Entrada};
        \draw[thick] (0,1) rectangle (2,1.8) node[pos=.5] {\scriptsize Embedding};
        \draw[thick] (0,2) rectangle (2,2.8) node[pos=.5] {\scriptsize Multi-Head};
        \draw[thick] (0,3) rectangle (2,3.8) node[pos=.5] {\scriptsize Atenção};
        \draw[thick] (0,4) rectangle (2,4.8) node[pos=.5] {\scriptsize Feed Forward};
        \draw[thick] (0,5) rectangle (2,5.8) node[pos=.5] {\scriptsize Saída};
        
        \draw[thick, ->] (1,0.8) -- (1,1);
        \draw[thick, ->] (1,1.8) -- (1,2);
        \draw[thick, ->] (1,2.8) -- (1,3);
        \draw[thick, ->] (1,3.8) -- (1,4);
        \draw[thick, ->] (1,4.8) -- (1,5);
      \end{tikzpicture}
    \end{column}
    \begin{column}{0.48\textwidth}
      \textbf{Mixture of Experts}
      \begin{itemize}
        \item Rede de roteamento com especialistas
        \item Eficiente - ativa só especialistas relevantes
        \item Adequado para multi-tarefas
      \end{itemize}
      
      \vspace{0.5cm}
      
      \centering
      \begin{tikzpicture}[scale=0.6]
        \draw[thick] (1,0) rectangle (1.5,0.8) node[pos=.5] {\scriptsize Dados};
        
        \draw[thick] (-0.5,2) rectangle (0,2.8) node[pos=.5] {\scriptsize E1};
        \draw[thick] (0.25,2) rectangle (0.75,2.8) node[pos=.5] {\scriptsize E2};
        \draw[thick] (1,2) rectangle (1.5,2.8) node[pos=.5] {\scriptsize E3};
        \draw[thick] (1.75,2) rectangle (2.25,2.8) node[pos=.5] {\scriptsize E4};
        \draw[thick] (2.5,2) rectangle (3,2.8) node[pos=.5] {\scriptsize En};
        
        \draw[thick] (0.5,1.2) rectangle (2,1.8) node[pos=.5] {\scriptsize Gating Network};
        \draw[thick] (0.75,4) rectangle (1.75,4.8) node[pos=.5] {\scriptsize Combinador};
        
        \draw[thick, ->] (1.25,0.8) -- (1.25,1.2);
        \draw[thick, ->] (1.25,1.8) -- (-0.25,2);
        \draw[thick, ->] (1.25,1.8) -- (0.5,2);
        \draw[thick, ->] (1.25,1.8) -- (1.25,2);
        \draw[thick, ->] (1.25,1.8) -- (2,2);
        \draw[thick, ->] (1.25,1.8) -- (2.75,2);
        
        \draw[thick, ->] (-0.25,2.8) -- (1,4);
        \draw[thick, ->] (0.5,2.8) -- (1.1,4);
        \draw[thick, ->] (1.25,2.8) -- (1.25,4);
        \draw[thick, ->] (2,2.8) -- (1.4,4);
        \draw[thick, ->] (2.75,2.8) -- (1.5,4);
      \end{tikzpicture}
    \end{column}
  \end{columns}
  
  \vspace{0.3cm}
  
  \begin{table}
    \centering
    \caption{Comparação de Desempenho no Dataset CALCE CS2}
    \begin{tabular}{lcc}
      \toprule
      \textbf{Método} & \textbf{RMSE} & \textbf{MAE} \\
      \midrule
      Transformer & 0.0847 & 0.0621 \\ 
      MoE & 0.0923 & 0.0701 
      \bottomrule
    \end{tabular>
  \end{table>
\end{frame}

\begin{frame}{Método Escolhido: TimesNet}
  \begin{block}{Porquê TimesNet?}
    \begin{itemize}
      \item 	extbf{Especialização}: Concebido especificamente para análise de séries temporais
      \item \textbf{Eficiência}: Melhor equilíbrio desempenho/computação que Transformers
      \item \textbf{Multi-periodicidade}: Deteta múltiplos padrões temporais simultaneamente
      \item \textbf{Transformação 2D}: Converte dados 1D em tensores 2D para processamento CNN
    \end{itemize}
  \end{block}
  
  \vspace{0.3cm}
  
  \begin{exampleblock}{Vantagens para Dados de Bateria}
    \begin{itemize}
      \item Captura padrões de curto prazo (ciclos de carga/descarga)
      \item Deteta tendências de longo prazo (degradação)
      \item Usa FFT para descoberta automática de periodicidades
    \end{itemize}
  \end{exampleblock}
  
  \vspace{0.3cm}
  
  \centering
  \begin{tikzpicture}[scale=0.8]
    \draw[thick] (0,0) rectangle (1.5,1) node[pos=.5] {Dados 1D};
    \draw[thick, ->] (1.7,0.5) -- (2.3,0.5) node[midway,above] {\scriptsize FFT};
    \draw[thick] (2.5,0) rectangle (4,1) node[pos=.5] {Tensor 2D};
    \draw[thick, ->] (4.2,0.5) -- (4.8,0.5);
    \draw[thick] (5,0) rectangle (6.5,1) node[pos=.5] {CNN 2D};
    \draw[thick, ->] (6.7,0.5) -- (7.3,0.5);
    \draw[thick] (7.5,0) rectangle (9,1) node[pos=.5] {Predição};
  \end{tikzpicture}
\end{frame>

\section{Dataset e Metodologia}
\begin{frame}{Dataset CALCE CS2}
  \begin{block}{Características do Dataset}
    \begin{itemize}
      \item \textbf{886 ciclos} de dados de bateria Li-ion
      \item Medições de tensão, corrente, temperatura
      \item Cálculos derivados de SOC, SOH e RUL
      \item Pré-processamento e normalização de dados
    \end{itemize}
  \end{block}
  
  \vspace{0.3cm>
  
  \begin{exampleblock}{Estrutura dos Dados}
    \begin{itemize}
      \item Organização hierárquica: batches → ciclos → medições
      \item Sistema de seleção de ciclos para treino/validação/teste
      \item Pipeline de pré-processamento automatizado
      \item Gestão de configurações para reprodutibilidade
    \end{itemize}
  \end{exampleblock}
\end{frame>

\begin{frame}{Abordagem Selecionada: TimesNet}
  \begin{block}{O que é o TimesNet?}
    \begin{itemize}
      \item Estado da arte para previsão de séries temporais
      \item Usa Transformada Rápida de Fourier (FFT) para detetar períodos
      \item Converte dados 1D em tensores 2D para análise CNN 2D
    \end{itemize}
  \end{block}
  
  \begin{exampleblock}{Porquê se adequa}
    Captura padrões de bateria de curto e longo prazo eficazmente
  \end{exampleblock}
  
  \vspace{0.5cm}
  
  \centering
  \begin{tikzpicture}[scale=0.8]
    % TimesNet architecture visualization
    \draw[thick] (0,0) rectangle (2,1) node[pos=.5] {Dados 1D};
    \draw[thick, ->] (2.2,0.5) -- (2.8,0.5) node[midway,above] {FFT};
    \draw[thick] (3,0) rectangle (5,1) node[pos=.5] {Tensor 2D};
    \draw[thick, ->] (5.2,0.5) -- (5.8,0.5);
    \draw[thick] (6,0) rectangle (8,1) node[pos=.5] {CNN};
  \end{tikzpicture}
\end{frame}

\begin{frame}{Metodologia}
  \begin{block}{Adaptação}
    \begin{itemize}
      \item \textbf{Entradas}: Tensão, corrente, temperatura, etc.
      \item \textbf{Alvos}: SOC, SOH, RUL
    \end{itemize}
  \end{block}
  
  \vspace{0.3cm}
  
  \textbf{Ferramentas Utilizadas}:
  \begin{itemize}
    \item Otimização de hiperparâmetros com \textbf{Optuna}
    \item Acompanhamento de experiências com \textbf{Weights \& Biases}
  \end{itemize}
  
  \vspace{0.5cm}
  
  \centering
  \begin{tikzpicture}[scale=0.7]
    % Methodology flowchart
    \draw[thick] (0,0) rectangle (2,0.8) node[pos=.5] {Dados de Entrada};
    \draw[thick, ->] (2.2,0.4) -- (2.8,0.4);
    \draw[thick] (3,0) rectangle (5,0.8) node[pos=.5] {TimesNet};
    \draw[thick, ->] (5.2,0.4) -- (5.8,0.4);
    \draw[thick] (6,0) rectangle (8,0.8) node[pos=.5] {Previsões};
    
    \node at (2.5,1.2) {Optuna};
    \node at (2.5,-0.5) {W\&B};
  \end{tikzpicture}
\end{frame}

\begin{frame}{Otimização de Hiperparâmetros}
  \begin{block}{Processo de Otimização com Optuna}
    \begin{itemize}
      \item 	extbf{50 tentativas} de otimização
      \item Tree-structured Parzen Estimator (TPE) para busca inteligente
      \item Monitorização em tempo real via Weights \& Biases
    \end{itemize}
  \end{block}
  
  \vspace{0.3cm}
  
  \begin{exampleblock}{Parâmetros Otimizados}
    \begin{itemize}
      \item Learning rate, batch size, dimensões dos embeddings
      \item Número de camadas, dropout rate
      \item Parâmetros específicos do TimesNet (top-k períodos)
    \end{itemize}
  \end{exampleblock}
  
  \vspace{0.5cm}
  
  \centering
  \begin{tikzpicture}[scale=0.8]
    \draw[thick, ->] (0,0) -- (6,0) node[right] {Tentativas};
    \draw[thick, ->] (0,0) -- (0,3) node[above] {MSE Loss};
    
    \draw[red, thick] (0,2.5) -- (1,2.3) -- (2,2.0) -- (3,1.6) -- (4,1.3) -- (5,1.0);
    
    
ode[red] at (3,3) {Convergência da Otimização};
  \end{tikzpicture}
\end{frame}

\section{Experiências e Resultados}
\begin{frame}{Configuração do Treino}
  \begin{block}{Configuração Final}
    \begin{itemize}
      \item \textbf{43 épocas} de treino (30,81 horas)
      \item Loss de validação final: \textbf{0,02929}
      \item Arquitetura com \textbf{2,4M parâmetros}
      \item Features de entrada: tensão, corrente, capacidade, etc.
    \end{itemize}
  \end{block}
  
  \vspace{0.3cm}
  
  \begin{exampleblock}{Divisão dos Dados}
    \begin{itemize}
      \item 70\% treino, 15\% validação, 15\% teste
      \item Estratégia de janela deslizante para séries temporais
      \item Normalização por feature para estabilidade numérica
    \end{itemize}
  \end{exampleblock}
\end{frame>

\begin{frame}{Experiências: Treino}
  \begin{block}{Configuração de Treino}
    \begin{itemize}
      \item 43 épocas, 30,81 horas de tempo de treino
      \item Perda de validação final: 0,02929
      \item Otimizado com Optuna (50 tentativas)
      \item Monitorizado via Weights \& Biases
    \end{itemize}
  \end{block}
  
  \vspace{0.5cm}
  
  \centering
  \begin{tikzpicture}[scale=0.8]
    % Training curve visualization
    \draw[thick, ->] (0,0) -- (5,0) node[right] {Épocas};
    \draw[thick, ->] (0,0) -- (0,3) node[above] {Perda};
    
    \draw[red, thick] (0,2.5) .. controls (1,2) and (2,1.5) .. (3,1) .. controls (4,0.8) .. (4.5,0.7);
    
    \node[red] at (2.5,3) {Perda de Validação};
  \end{tikzpicture}
\end{frame}

\section{Resultados}
\begin{frame}{Resultados de Predição}
  \begin{table}
    \centering
    \caption{Métricas de Desempenho por Parâmetro}
    \begin{tabular}{lccc}
      	oprule
      	extbf{Parâmetro} & 	extbf{RMSE} & 	extbf{MAE} & 	extbf{MAPE (\%)} 
      \midrule
      SOC & 	extbf{0,2251} & 	extbf{0,1234} & 	extbf{2,45} 
      SOH & 0,5252 & 0,3456 & 4,78 
      RUL & 0,5311 & 0,3623 & 5,12 
      \bottomrule
    \end{tabular}
  \end{table}
  
  \vspace{0.3cm}
  
  \begin{exampleblock}{Análise dos Resultados}
    \begin{itemize}
      \item 	extbf{SOC}: Melhor desempenho - variações mais previsíveis
      \item 	extbf{SOH}: Moderado - degradação gradual complexa
      \item 	extbf{RUL}: Mais desafiante - previsão a longo prazo
    \end{itemize}
  \end{exampleblock}
\end{frame}

\begin{frame}{Visualização das Predições - SOC}
  \begin{center}
    	extbf{Comparação: Valores Reais vs. Preditos (SOC)}
  \end{center>
  
  \vspace{0.5cm}
  
  \centering
  \begin{tikzpicture}[scale=0.9]
    \draw[thick, ->] (0,0) -- (8,0) node[right] {Tempo (ciclos)};
    \draw[thick, ->] (0,0) -- (0,4) node[above] {SOC (\%)};
    
    % Valores reais (linha azul)
    \draw[blue, thick] (0,1) -- (1,2.5) -- (2,1.2) -- (3,3) -- (4,1.5) -- (5,2.8) -- (6,1.8) -- (7,2.2);
    
    % Valores preditos (linha vermelha tracejada)
    \draw[red, dashed, very thick] (0,1.1) -- (1,2.4) -- (2,1.3) -- (3,2.9) -- (4,1.6) -- (5,2.7) -- (6,1.9) -- (7,2.1);
    
    \legend{blue}{Valores Reais}
    \legend{red}{Valores Preditos}
    
    
ode[blue, above] at (2,3.5) {Real};
    
ode[red, above] at (5,3.5) {Predito};
  \end{tikzpicture}
  
  \vspace{0.3cm}
  
  \begin{alertblock}{Observações}
    Excelente concordância entre valores reais e preditos, especialmente em transições de carga/descarga
  \end{alertblock}
\end{frame>

\begin{frame}{Resultados: Visualizações}
  \begin{itemize}
    \item \textbf{Previsão SOC}: Acompanha de perto os valores reais
    \item \textbf{SOH/RUL}: Precisão moderada, margem para melhoria
  \end{itemize}
  
  \vspace{0.5cm}
  
  \centering
  \begin{tikzpicture}[scale=0.8]
    % Prediction vs actual visualization
    \draw[thick, ->] (0,0) -- (6,0) node[right] {Tempo};
    \draw[thick, ->] (0,0) -- (0,3) node[above] {SOC};
    
    \draw[blue, thick] (0,1) sin (1,2) cos (2,1.5) sin (3,2.2) cos (4,1.8) sin (5,2);
    \draw[red, dashed, thick] (0,1.1) sin (1,1.9) cos (2,1.6) sin (3,2.1) cos (4,1.7) sin (5,1.9);
    
    \node[blue] at (1,2.7) {Real};
    \node[red] at (3,2.7) {Previsto};
  \end{tikzpicture}
\end{frame}

\section{Discussão e Análise}
\begin{frame}{Análise de Desempenho}
  \begin{block}{Pontos Fortes}
    \begin{itemize}
      \item 	extbf{SOC}: Excelente precisão (RMSE: 0,2251)
      \item 	extbf{Adaptação TimesNet}: Bem-sucedida para dados de bateria
      \item 	extbf{Padrões Multi-periódicos}: FFT identifica múltiplas periodicidades
    \end{itemize}
  \end{block}
  
  \vspace{0.3cm}
  
  \begin{alertblock}{Desafios Identificados}
    \begin{itemize}
      \item 	extbf{Aprendizagem Multi-tarefa}: Dilui performance vs. modelos especializados
      \item 	extbf{RUL}: Predição a longo prazo mais complexa
      \item 	extbf{Complexidade}: 2,4M parâmetros limitam implementação em tempo real
    \end{itemize}
  \end{alertblock}
  
  \vspace{0.3cm}
  
  \begin{exampleblock}{Comparação com Métodos Simples}
    TimesNet supera métodos baseline, mas trade-off complexidade vs. ganho marginal
  \end{exampleblock}
\end{frame>

\begin{frame}{Limitações e Insights}
  \begin{alertblock}{Limitações do Estudo}
    \begin{itemize}
      \item Dataset único (CALCE CS2) - generalização limitada
      \item Condições laboratoriais vs. aplicações reais
      \item Químicas de bateria limitadas (apenas Li-ion)
    \end{itemize}
  \end{alertblock}
  
  \vspace{0.3cm}
  
  \begin{block}{Insights Metodológicos}
    \begin{itemize}
      \item Aprendizagem multi-tarefa requer arquiteturas específicas
      \item FFT crucial para deteção automática de periodicidades
      \item Otimização de hiperparâmetros essencial para convergência
    \end{itemize}
  \end{block}
  
  \vspace{0.3cm}
  
  \centering
  \begin{tikzpicture}[scale=0.7]
    \draw[thick] (0,1) rectangle (2,2) node[pos=.5] {Multi-tarefa};
    \draw[thick] (4,0.5) rectangle (6,1.5) node[pos=.5] {SOC};
    \draw[thick] (4,1.5) rectangle (6,2.5) node[pos=.5] {SOH};
    \draw[thick] (4,2.5) rectangle (6,3.5) node[pos=.5] {RUL};
    
    \draw[thick, <->] (2.2,1.5) -- (3.8,2);
    
ode at (3,1) {Trade-off};
  \end{tikzpicture>
\end{frame}

\section{Conclusões e Trabalho Futuro}
\begin{frame}{Principais Contribuições}
  \begin{block}{Contribuições Científicas}
    \begin{itemize}
      \item 	extbf{Primeira adaptação} do TimesNet para monitorização de saúde de baterias
      \item 	extbf{Análise comparativa} completa: Transformer vs. MoE vs. TimesNet
      \item 	extbf{Estudo dos trade-offs} da aprendizagem multi-tarefa em baterias
    \end{itemize}
  \end{block}
  
  \vspace{0.3cm}
  
  \begin{exampleblock}{Resultados Práticos}
    \begin{itemize}
      \item SOC com precisão de 97,5\% (RMSE: 0,2251)
      \item Deteção automática de padrões multi-periódicos
      \item Framework reprodutível com Optuna + Weights \& Biases
    \end{itemize}
  \end{exampleblock}
  
  \vspace{0.3cm}
  
  \begin{alertblock}{Limitações Reconhecidas}
    Complexidade computacional vs. ganhos marginais em RUL/SOH
  \end{alertblock}
\end{frame>

\begin{frame>{Direções Futuras}
  \begin{block}{Direções de Investigação}
    \begin{itemize}
      \item 	extbf{Modelos Especializados}: Uma rede para cada parâmetro (SOC/SOH/RUL)
      \item 	extbf{Abordagens Híbridas}: Combinação IA + modelos físicos
      \item 	extbf{Implementação Real}: Otimização para sistemas embebidos
    \end{itemize}
  \end{block}
  
  \vspace{0.3cm}
  
  \begin{exampleblock}{Extensões do Trabalho}
    \begin{itemize}
      \item 	extbf{Datasets Diversos}: Múltiplas químicas e condições
      \item 	extbf{Transfer Learning}: Adaptação entre diferentes baterias
      \item 	extbf{Incerteza}: Quantificação de incerteza nas predições
      \item 	extbf{Aplicações Reais}: Veículos elétricos e sistemas ferroviários
    \end{itemize}
  \end{exampleblock}
\end{frame>\begin{frame}[plain]
  \centering
  \Huge 	extbf{Obrigado!}
  
  \vspace{1cm}
  
  \Large 	extbf{BattAIHealth}
  
  
ormalsize
  	extbf{Battery Condition Estimation in Automotive and Railway Applications Using AI}
  
  \vspace{1cm}
  
  
ormalsize
  Perguntas e Discussão
  
  \vspace{0.5cm}
  \small
  Pedro André Silva Ferreira
  Projeto Final - Engenharia Eletrotécnica e de Computadores
  Instituto Politécnico de Leiria
  Julho 2025
\end{frame>

% Bibliography
\begin{frame}{Referências Principais}
  \begin{thebibliography}{9}
    \small
    \bibitem{timesnet}
      Haixu Wu, Tengge Hu, Yong Liu, et al.
      \emph{TimesNet: Temporal 2D-Variation Modeling for General Time Series Analysis}.
      ICLR, 2023.
    
    \bibitem{calce}
      CALCE Battery Research Group.
      \emph{Lithium-ion Battery Experimental Data}.
      University of Maryland.
      
    \bibitem{optuna}
      Takuya Akiba, Shotaro Sano, et al.
      \emph{Optuna: A Next-generation Hyperparameter Optimization Framework}.
      KDD, 2019.
      
    \bibitem{transformer}
      Ashish Vaswani, Noam Shazeer, et al.
      \emph{Attention Is All You Need}.
      NIPS, 2017.
      
    \bibitem{moe}
      Noam Shazeer, Azalia Mirhoseini, et al.
      \emph{Outrageously Large Neural Networks: The Sparsely-Gated Mixture-of-Experts Layer}.
      ICLR, 2017.
  \end{thebibliography}
\end{frame}

\end{document}

% Bibliography
\begin{frame}{Referências}
  \begin{thebibliography}{9}
    \bibitem{timesnet}
      Haixu Wu, Tengge Hu, Yong Liu, Hang Zhou, Jianmin Wang, Mingsheng Long.
      \emph{TimesNet: Temporal 2D-Variation Modeling for General Time Series Analysis}.
      ICLR, 2023.
    
    \bibitem{calce}
      CALCE Battery Research Group.
      \emph{Lithium-ion Battery Experimental Data}.
      University of Maryland.
      
    \bibitem{optuna}
      Takuya Akiba, Shotaro Sano, Tetsuya Teshima, Takeru Ohta, Masanori Koyama.
      \emph{Optuna: A Next-generation Hyperparameter Optimization Framework}.
      KDD, 2019.
  \end{thebibliography}
\end{frame}

\end{document}
